\documentclass{report}
% PACKAGES
\usepackage[utf8]{inputenc}
\usepackage{mathtools} % math and figures
\usepackage{float} % make figure appear where we want with [H]
\usepackage{filecontents}
\usepackage[numbered,framed]{matlab-prettifier}
% these packages include more math symbols you might use
\usepackage{amsmath,amsfonts,amsthm,amssymb}


% PROJECT Specific Information to Fill Out
\newcommand{\LectureTitle}{Empirical Asset Pricing}
\newcommand{\LectureDate}{\today}
\newcommand{\LectureClassName}{ECON676}
\newcommand{\LatexerName}{Wanxin Chen}
\author{\LatexerName}


% CONFIGURATIONS to make the report look better
\usepackage{setspace}
\usepackage{Tabbing}
\usepackage{fancyhdr}
\usepackage{lastpage}
\usepackage{extramarks}
\usepackage{afterpage}
\usepackage{abstract}

% In case you need to adjust margins:
\topmargin=-0.45in
\evensidemargin=0in
\oddsidemargin=0in
\textwidth=6.5in
\textheight=9.0in
\headsep=0.25in

% Setup the header and footer
\pagestyle{fancy}
\lhead{\LatexerName}
\chead{\LectureClassName: \LectureTitle}
\rhead{\LectureDate}
\lfoot{\lastxmark}
\cfoot{}
\rfoot{Page\ \thepage\ of\ \pageref{LastPage}}
\renewcommand\headrulewidth{0.4pt}
\renewcommand\footrulewidth{0.4pt}
\usepackage{booktabs}

\title{\LectureTitle: Problem Set 6}

\begin{document}
\maketitle
\newpage

\subsection{a}
The table below shows the time-series average return, t-statistic, annualized Sharpe ratio and standard deviation on this 1-month, 1-month industry momentum portfolio.
\begin{table}[H]
\centering
\begin{tabular}{|l|l|l|l|l|}
\hline
                        & Average return & t-statistic & Annualized Sharpe ratio & Standard deviation \\ \hline
1-month,1-month IND MOM & 0.6936         & 4.1919      & 0.4411                  & 5.4477             \\ \hline
\end{tabular}
\end{table}

\subsection{b}
For the 1-month, 1-month strategy, we decompose its returns as follows:
\[ Mom = \sigma_{\mu}^{2} + \sigma_{\beta}^{2}cov(\tilde{F}_{t},\tilde{F}_{t-1}) + \frac{1}{N}\sum_{j=1}^{N}cov( \epsilon_{j,t},\epsilon_{j,t-1}) \]
Here, we used CAPM to get the $\mu$, $\beta$ and $\epsilon$ as follows:
\[ r_{it}-r_{ft} = \alpha_{i} + \beta_{i}(r_{mt}-r_{ft}) + \epsilon_{it} \]
By comparing the number in the table below, we find out that the third term, $\frac{1}{N}\sum_{j=1}^{N}cov( \epsilon_{j,t},\epsilon_{j,t-1})$, which is the serial covariation in firm-specific components, is the greatest contributor to momentum profits.
\begin{table}[H]
\centering
\begin{tabular}{|l|l|l|l|}
\hline
    & $ \sigma_{\mu}^{2}$ & $ \sigma_{\beta}^{2}cov(\tilde{F}_{t},\tilde{F}_{t-1})$ & $\frac{1}{N}\sum_{j=1}^{N}cov( \epsilon_{j,t},\epsilon_{j,t-1})$ \\ \hline
Mom & 0.0122         & 0.1337      & 0.6351                  \\ \hline
\end{tabular}
\end{table}

\subsection{c}
The table below shows the time-series average return, t-statistic, annualized Sharpe ratio and standard deviation on this 12-month, 1-month industry momentum portfolio.
\begin{table}[H]
\begin{tabular}{|l|l|l|l|l|}
\hline
                        & Average return & t-statistic & Annualized Sharpe ratio & Standard deviation \\ \hline
12-month,1-month IND MOM & 0.8583         & 4.5857   & 0.4849               & 6.1310            \\ \hline
\end{tabular}
\end{table}

\subsection{d}
The table below shows the time-series average return, t-statistic, annualized Sharpe ratio and standard deviation on this 12-month, 1-month, skip 1 month industry momentum portfolio and differences between this strategy and the 12-month, 1-month, no skipping momentum strategy above. From the table, we can see this strategy achieves $0.0701$ percent more average monthly returns, higher t-statistics, $0.0414$ bigger annual Sharpe ratio and $0.0200$ lower standard deviation. The difference shows the 12-month, 1-month, skip 1-month momentum strategy performs better than 12-month, 1-month, skip 1 month momentum strategy.
\begin{table}[H]
\begin{tabular}{|l|l|l|l|l|}
\hline
                        & Average return & t-statistic & Annualized Sharpe ratio & Standard deviation \\ \hline
12-1 month,skip 1 month IND MOM & 0.9284        & 4.9765     & 0.5263               & 6.1110             \\ \hline
12-1 month, no skip IND MOM & 0.8583         & 4.5857   & 0.4849               & 6.1310            \\ \hline
Difference & 0.0701         & 0.3908   & 0.0414         & -0.0200            \\ \hline
\end{tabular}
\end{table}

\subsection{e}
The table below shows the alpha and betas of three momentum strategies using the following regression,
\[ r_{MOM,t} = \alpha + \beta_{RMRF}( r_{mt}-r_{ft}) + \beta_{SMB} r_{SMB,t} + \beta_{HML} r_{HML,t} + \epsilon_{t} \]
From the table, we can see the $\alpha$ of three regressions are all quite big and apparently different from zero. The t-statistic of $\alpha$s are larger than 1.96, so we can reject the null hypothesis that $\alpha = 0$ at 5\% significance level.  Thus, the Fama-French model can not price the momentum strategies.
\begin{table}[H]
\centering
\begin{tabular}{|l|l|l|l|l|l|}
\hline
                           & $\alpha$  & $t(\alpha)$ & $\beta_{RMRF}$    & $\beta_{SMB}$     & $\beta_{HML}$     \\ \hline
1-1 month IND MOM          & 0.6924 & 4.1527 & -0.0634 & -0.0514 & 0.1329  \\ \hline
12-1 month,no skip,IND MOM & 1.0449 & 5.6479 & -0.1365 & 0.0509  & -0.2796 \\ \hline
12-1 month,skip 1 month,IND MOM    & 1.1076 & 6.0057 & -0.1321 & 0.0845  & -0.2866 \\ \hline
\end{tabular}
\end{table}

\subsection{f}
The table below shows the alpha and betas of three momentum strategies using the following regression,
\[ r_{MOM,t} = \alpha + \beta_{RMRF}( r_{mt}-r_{ft}) + \beta_{SMB} r_{SMB,t} + \beta_{HML} r_{HML,t} + \beta_{UMD} r_{UMD}+ \epsilon_{t} \]
From the table, we can see that $\alpha$ for 1-month, 1-month momentum strategy is $0.5631$ and t-statistics is 3.2937, which is  still bigger than 1.96. So we can reject the null hypothesis that $\alpha = 0$ at 5\% significance level and the four factor model stiil cannot explain the 1-month, 1-month strategy. 

Also, we notice the $\alpha$ of other two regressions are small. The t-statistic of $\alpha$s are smaller than 1.96, so we can not reject the null hypothesis that $\alpha = 0$ at 5\% significance level. Thus, the four factor model can price the intermediate-term and long-term momentum strategies now. Thus, we learn that the momentum factor UMD can well explain the intermediate-term and long-term momentum strategies returns but it still cannot explain the short-term momentum strategy.
\begin{table}[H]
\centering
\begin{tabular}{|l|l|l|l|l|l|l|}
\hline
                           & $\alpha$  & $t(\alpha)$ & $\beta_{RMRF}$    & $\beta_{SMB}$     & $\beta_{HML}$ & $\beta_{UMD}$   \\ \hline
1-1 month IND MOM          & 0.5631 & 3.2937 & -0.0318 & -0.0488 & 0.2003 & 0.1387 \\ \hline
12-1 month,no skip,IND MOM & 0.0785 & 0.5606 & 0.0747  & 0.0834  & 0.1859 & 0.9751 \\ \hline
12-1 month,skip,IND MOM    & 0.1428 & 1.0248 & 0.0787  & 0.1169  & 0.1781 & 0.9734 \\ \hline
\end{tabular}
\end{table}

According to the previous problem sets and this problem set, we conclude that cross section of returns can partly be explained by market, SMB, HML, UMD factors.  Also, all the four factors
\end{document}

